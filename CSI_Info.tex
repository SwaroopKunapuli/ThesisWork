\section{Current status and information for the preparation of CSI}
\label{sec:csi_info}
This section summarizes the data (results and discussion) of the work done in the academic year 2023-2024, to be presented in the CSI-2024. 
The summary is divided into two categories: results and discussion that is already available and the work that is currently being done.
\subsection{Results and discussion of the work (academic year 2023-2024)}
The following is the list of distinct projects that were undertaken in the academic year 2023-2024:
\begin{enumerate}
    \item \textbf{Developing the algorithm for fragmentation simulations of non-covalent clusters}
    \item Studying the stability and water retention properties of the proposed NADES systems in lichen EverniaPrunastri.
    \item Benchmarking the workflow for Metadynamics simulations at semi-empirical and DFT levels of theory.
    \item Going beyond the thermochemical calculations, Local Energy Decomposition Analysis (LEDA) as an attempt for explaining the stability of certain non-covalent clusters observed in the experiment.
\end{enumerate}
The following part of this project starts with the section \ref{sec:lit_review} on literature review for the above listed works. The section \ref{} explains our approach to the fragmentation simulations and attempts to solve the computational bottleneck for these simulations.
The section \ref{sec:meta_benchmark} gives some results on benchmarking on different QM methods that are planned to be employed in the fragmentation simulations.
The section \ref{sec:benchmarking_dfa} gives the results of the benchmarking of different density functional approximations for the geometry optimization calculations.
The section \ref{sec:led_analysis} gives the results of the LEDA calculations for the non-covalent clusters.
The section \ref{} introduces the lichen system \textit{Evernia prunasitri} and the experimental MS data to understand their water-retention properties via possible formation of NADES. The results of the thermochemical calculations of different combinations of clusters are presented here.
