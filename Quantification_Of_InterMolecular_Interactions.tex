\section{Quantification of intermolecular interactions}
\label{sec:quantification_interactions}
In this section we will discuss the methods and results of the calculations to quantify the strength of the intermolecular interactions in our choline chloride-based systems.



The Local Energy Decomposition method is implemented in ORCA to decompose the DLPNO-CCSD(T) energies into physically meaningful contributions.
Admittedly, the decomposition is to some extent arbitary, but using pair correlation energies for the pairs of localized occupied orbitals help in decomposing the correlation and the reference energy into intra- and interfragment contributions.
The fragments have to be user defined, and in our case these are the urea,thiourea and choline molecules and the chloride ions.
We studied the following urea-containing and the corresponding thiourea-containing clusters.
For each cluster combinations, we studied three geometries:
\begin{table}
\centering
\begin{tabular}{|c|c|c|c|c|c|c|} \hline
System & Old Geometry & New Geometry & Crest Optimised Geometry \\ \hline
$[1U+1ChCl+1Cl]^{-1}$ &  & &  \\ \hline
$[1ThioU+1ChCl+1Cl]^{-1}$ &  & &  \\ \hline
\end{tabular}
\caption{Interaction energies of three geometries of each cluster combination studied}
\end{table}
\subsection{Input structure and parameters for different Calculations}
\subsubsection{NBO Analysis}
GAUSSIAN 16 Package for the NBO population analysis
\begin{minted}[linenos=false]{bash}
    #P B3LYP/6-311++G(d,p) SCF=Tight Pop=NBO
\end{minted}
\subsubsection{QTAIM Analysis}
WFN files were generated using GAUSSIAN 16 Package.
\begin{minted}[linenos=false]{bash}
    #P B3LYP/6-311++G(d,p) SCF=Tight output=WFN
    ...
    ...
    <filename>.wfn
\end{minted}
\subsubsection{Local Energy Decomposition Analysis}
ORCA 5.0.4 Package was used for the Local Energy Decomposition (LED) Analysis.

\begin{minted}[linenos=false]{bash}
! dlpno-ccsd(t) def2-TZVP def2-TZVP/C def2/J rijcosx verytightscf 
! TightPNO LED
%mdci DoDIDplot true end
%MaxCore 8000
%pal nprocs 32 end
*xyz -1 1
... 
*
\end{minted}

Resolution of identity (RI) method with Chain of Spheres (COS) approximation (RIJCOSX) implemented in the ORCA package was used to speed up the calculations,
along with DLPNO-CCSD(T) method and auxiliary basis set def2-TZVP/C.
The calculations requested data for DIDplot which will be presented in this report.
