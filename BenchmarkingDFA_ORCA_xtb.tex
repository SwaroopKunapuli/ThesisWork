\section{Benchmarking density functional approximations}
List of available methods and codes for speeding up the optimization calculations:
\begin{enumerate}
    \item ORCA (DFA and composite methods)
    \item CENSO (with ORCA as the QM calculation driver and xTB as the semi-empirical method; parameters already benchmarked by Grimme's lab)
    \item ML-based methods (AIQM1 method in MLatom 3)
\end{enumerate}
Eventhough ORCA has its own optimization algorithm, even with approximations included the calculations are long.
xTB gives a way to do optimization on a fixed number of cycles while calculating single-point energies with ORCA. This is what CENSO does but can also be done with xTB and ORCA alone.
\begin{table}
\centering

\begin{tabular}{|c|c|c|c|} \hline
Directory & Orca settings & xTB settings & time(min) \\ \hline
20 &  &  & \\ \hline
19 &  &  & \\ \hline
18 &  &  & \\ \hline
17 &  &  & \\ \hline

\end{tabular}
\caption{Various approaches for optimization with ORCA/CENSO and the corresponding computation times on the 32-core Intel Cascadelake machine.}
\end{table}